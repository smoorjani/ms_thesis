% ****************************************************************************************************
% classicthesis-config.tex
% formerly known as loadpackages.sty, classicthesis-ldpkg.sty, and classicthesis-preamble.sty
% Use it at the beginning of your ClassicThesis.tex, or as a LaTeX Preamble
% in your ClassicThesis.{tex,lyx} with % ****************************************************************************************************
% classicthesis-config.tex
% formerly known as loadpackages.sty, classicthesis-ldpkg.sty, and classicthesis-preamble.sty
% Use it at the beginning of your ClassicThesis.tex, or as a LaTeX Preamble
% in your ClassicThesis.{tex,lyx} with % ****************************************************************************************************
% classicthesis-config.tex
% formerly known as loadpackages.sty, classicthesis-ldpkg.sty, and classicthesis-preamble.sty
% Use it at the beginning of your ClassicThesis.tex, or as a LaTeX Preamble
% in your ClassicThesis.{tex,lyx} with % ****************************************************************************************************
% classicthesis-config.tex
% formerly known as loadpackages.sty, classicthesis-ldpkg.sty, and classicthesis-preamble.sty
% Use it at the beginning of your ClassicThesis.tex, or as a LaTeX Preamble
% in your ClassicThesis.{tex,lyx} with \input{classicthesis-config}
% ****************************************************************************************************
% If you like the classicthesis, then I would appreciate a postcard.
% My address can be found in the file ClassicThesis.pdf. A collection
% of the postcards I received so far is available online at
% http://postcards.miede.de
% ****************************************************************************************************


% ****************************************************************************************************
% 0. Set the encoding of your files. UTF-8 is the only sensible encoding nowadays. If you can't read
% äöüßáéçèê∂åëæƒÏ€ then change the encoding setting in your editor, not the line below. If your editor
% does not support utf8 use another editor!
% ****************************************************************************************************
\PassOptionsToPackage{utf8}{inputenc}
  \usepackage{inputenc}

\PassOptionsToPackage{T1}{fontenc} % T2A for cyrillics
  \usepackage{fontenc}


% ****************************************************************************************************
% 1. Configure classicthesis for your needs here, e.g., remove "drafting" below
% in order to deactivate the time-stamp on the pages
% (see ClassicThesis.pdf for more information):
% ****************************************************************************************************
\PassOptionsToPackage{
  drafting=true,    % print version information on the bottom of the pages
  tocaligned=false, % the left column of the toc will be aligned (no indentation)
  dottedtoc=false,  % page numbers in ToC flushed right
  eulerchapternumbers=true, % use AMS Euler for chapter font (otherwise Palatino)
  linedheaders=false,       % chaper headers will have line above and beneath
  floatperchapter=true,     % numbering per chapter for all floats (i.e., Figure 1.1)
  eulermath=false,  % use awesome Euler fonts for mathematical formulae (only with pdfLaTeX)
  beramono=true,    % toggle a nice monospaced font (w/ bold)
  palatino=false,    % deactivate standard font for loading another one, see the last section at the end of this file for suggestions
  style=classicthesis % classicthesis, arsclassica
}{classicthesis}


% ****************************************************************************************************
% 2. Personal data and user ad-hoc commands (insert your own data here)
% ****************************************************************************************************
\newcommand{\myTitle}{Controllable Natural Langauge Generation for Audience-Centric Styles\xspace}
\newcommand{\mySubtitle}{An Homage to The Elements of Typographic Style\xspace}
\newcommand{\myDegree}{\textsc{Thesis}\\
Submitted in partial fulfillment of the requirements \\
for the degree of Master of Science in Computer Science\\ in the Graduate College of the \\ University of Illinois at Urbana-Champaign, 2023\xspace}
\newcommand{\myName}{Samraj Moorjani\xspace}
\newcommand{\myProf}{Hari Sundaram\xspace}
\newcommand{\myOtherProf}{Put name here\xspace}
\newcommand{\mySupervisor}{Put name here\xspace}
\newcommand{\myFaculty}{\noindent
% \noindent \textsc{Doctoral Committee}:\\
  % \indent Associate Professor Hari Sundaram, Chair, \\
  % \indent Assistant Professor Alexander Schwing, \\
  % \indent Professor ChengXiang Zhai,\\
  % \indent Dr. Chris Brew
  % \xspace
  }
\newcommand{\myDepartment}{Department of Computer Science\xspace}
\newcommand{\myUni}{University of Illinois, Urbana-Champaign\xspace}
\newcommand{\myLocation}{Urbana, Illinois \xspace}
\newcommand{\myTime}{May 2023\xspace}
\newcommand{\myVersion}{\classicthesis}

% ********************************************************************
% Setup, finetuning, and useful commands
% ********************************************************************
\providecommand{\mLyX}{L\kern-.1667em\lower.25em\hbox{Y}\kern-.125emX\@}
\newcommand{\ie}{i.\,e.}
\newcommand{\Ie}{I.\,e.}
\newcommand{\eg}{e.\,g.}
\newcommand{\Eg}{E.\,g.}
% ****************************************************************************************************


% ****************************************************************************************************
% 3. Loading some handy packages
% ****************************************************************************************************
% ********************************************************************
% Packages with options that might require adjustments
% ********************************************************************
\PassOptionsToPackage{ngerman,american}{babel} % change this to your language(s), main language last
% Spanish languages need extra options in order to work with this template
%\PassOptionsToPackage{spanish,es-lcroman}{babel}
    \usepackage{babel}

\DeclareUnicodeCharacter{0301}{HERE}
\usepackage{csquotes}
\PassOptionsToPackage{%
  backend=biber,
  bibencoding=utf8, %instead of bibtex
  % backend=bibtex8,
  % bibencoding=ascii,%
  doi=true,
  url=false,
  language=auto,%
  % style=numeric-comp,%
  style=authoryear-comp, % Author 1999, 2010
  %bibstyle=authoryear,dashed=false, % dashed: substitute rep. author with ---
  sorting=nyt, % name, year, title
  maxbibnames=10, % default: 3, et al.
  backref=true,%
  natbib=true % natbib compatibility mode (\citep and \citet still work)
}{biblatex}
    \usepackage{biblatex}

% avoids the too many math alphabets error
% https://www.texfaq.org/FAQ-manymathalph.html
% \newcommand\hmmax{0} % default is 3, restricts the number math groups; TeX doesn't allow more than 16
\newcommand\bmmax{2} % default is 4

\PassOptionsToPackage{fleqn}{amsmath}       % math environments and more by the AMS
  \usepackage{amsmath}

% ********************************************************************
% General useful packages
% ********************************************************************
\usepackage{graphicx} %
\usepackage{scrhack} % fix warnings when using KOMA with listings package
\usepackage{xspace} % to get the spacing after macros right
\PassOptionsToPackage{printonlyused,smaller}{acronym}
  \usepackage{acronym} % nice macros for handling all acronyms in the thesis
  %\renewcommand{\bflabel}[1]{{#1}\hfill} % fix the list of acronyms --> no longer working
  %\renewcommand*{\acsfont}[1]{\textsc{#1}}
  %\renewcommand*{\aclabelfont}[1]{\acsfont{#1}}
  %\def\bflabel#1{{#1\hfill}}
  \def\bflabel#1{{\acsfont{#1}\hfill}}
  \def\aclabelfont#1{\acsfont{#1}}
% ****************************************************************************************************
%\usepackage{pgfplots} % External TikZ/PGF support (thanks to Andreas Nautsch)
%\usetikzlibrary{external}
%\tikzexternalize[mode=list and make, prefix=ext-tikz/]
% ****************************************************************************************************


% ****************************************************************************************************
% 4. Setup floats: tables, (sub)figures, and captions
% ****************************************************************************************************
\usepackage{tabularx} % better tables
  \setlength{\extrarowheight}{3pt} % increase table row height
\newcommand{\tableheadline}[1]{\multicolumn{1}{l}{\spacedlowsmallcaps{#1}}}
\newcommand{\myfloatalign}{\centering} % to be used with each float for alignment
% \usepackage{side}

\usepackage[counterclockwise]{rotating}

% \usepackage{sidenotes}
% \usepackage{subfig} % clash with subcaption
% ****************************************************************************************************


% ****************************************************************************************************
% 5. Setup code listings
% ****************************************************************************************************
\usepackage{listings}
%\lstset{emph={trueIndex,root},emphstyle=\color{BlueViolet}}%\underbar} % for special keywords
\lstset{language=[LaTeX]Tex,%C++,
  morekeywords={PassOptionsToPackage,selectlanguage},
  keywordstyle=\color{RoyalBlue},%\bfseries,
  basicstyle=\small\ttfamily,
  %identifierstyle=\color{NavyBlue},
  commentstyle=\color{Green}\ttfamily,
  stringstyle=\rmfamily,
  numbers=none,%left,%
  numberstyle=\scriptsize,%\tiny
  stepnumber=5,
  numbersep=8pt,
  showstringspaces=false,
  breaklines=true,
  %frameround=ftff,
  %frame=single,
  belowcaptionskip=.75\baselineskip
  %frame=L
}
% ****************************************************************************************************

% Yes, classicthesis loads microtype with the option pdfspacing. You can change the options by issuing

\PassOptionsToPackage{kerning=true, tracking=false}{microtype}


% ****************************************************************************************************
% 6. Last calls before the bar closes
% ****************************************************************************************************
% ********************************************************************
% Her Majesty herself
% ********************************************************************
\usepackage{classicthesis}


% ********************************************************************
% Fine-tune hyperreferences (hyperref should be called last)
% ********************************************************************
\hypersetup{%
  %draft, % hyperref's draft mode, for printing see below
  colorlinks=true, linktocpage=true, pdfstartpage=3, pdfstartview=FitV,%
  % uncomment the following line if you want to have black links (e.g., for printing)
  %colorlinks=false, linktocpage=false, pdfstartpage=3, pdfstartview=FitV, pdfborder={0 0 0},%
  breaklinks=true, pageanchor=true,%
  pdfpagemode=UseNone, %
  % pdfpagemode=UseOutlines,%
  plainpages=false, bookmarksnumbered, bookmarksopen=true, bookmarksopenlevel=1,%
  hypertexnames=true, pdfhighlight=/O,%nesting=true,%frenchlinks,%
  urlcolor=CTurl, linkcolor=CTlink, citecolor=CTcitation, %pagecolor=RoyalBlue,%
  %urlcolor=Black, linkcolor=Black, citecolor=Black, %pagecolor=Black,%
  pdftitle={\myTitle},%
  pdfauthor={\textcopyright\ \myName, \myUni, \myFaculty},%
  pdfsubject={},%
  pdfkeywords={},%
  pdfcreator={pdfLaTeX},%
  pdfproducer={LaTeX with hyperref and classicthesis}%
}


% ********************************************************************
% Setup autoreferences (hyperref and babel)
% ********************************************************************
% There are some issues regarding autorefnames
% http://www.tex.ac.uk/cgi-bin/texfaq2html?label=latexwords
% you have to redefine the macros for the
% language you use, e.g., american, ngerman
% (as chosen when loading babel/AtBeginDocument)
% ********************************************************************
\makeatletter
\@ifpackageloaded{babel}%
  {%
    \addto\extrasamerican{%
      \renewcommand*{\figureautorefname}{Figure}%
      \renewcommand*{\tableautorefname}{Table}%
      \renewcommand*{\partautorefname}{Part}%
      \renewcommand*{\chapterautorefname}{Chapter}%
      \renewcommand*{\sectionautorefname}{Section}%
      \renewcommand*{\subsectionautorefname}{Section}%
      \renewcommand*{\subsubsectionautorefname}{Section}%
    }%
    \addto\extrasngerman{%
      \renewcommand*{\paragraphautorefname}{Absatz}%
      \renewcommand*{\subparagraphautorefname}{Unterabsatz}%
      \renewcommand*{\footnoteautorefname}{Fu\"snote}%
      \renewcommand*{\FancyVerbLineautorefname}{Zeile}%
      \renewcommand*{\theoremautorefname}{Theorem}%
      \renewcommand*{\appendixautorefname}{Anhang}%
      \renewcommand*{\equationautorefname}{Gleichung}%
      \renewcommand*{\itemautorefname}{Punkt}%
    }%
      % Fix to getting autorefs for subfigures right (thanks to Belinda Vogt for changing the definition)
      \providecommand{\subfigureautorefname}{\figureautorefname}%
    }{\relax}
\makeatother


% ********************************************************************
% Development Stuff
% ********************************************************************
\listfiles
%\PassOptionsToPackage{l2tabu,orthodox,abort}{nag}
%  \usepackage{nag}
%\PassOptionsToPackage{warning, all}{onlyamsmath}
%  \usepackage{onlyamsmath}


% ****************************************************************************************************
% 7. Further adjustments (experimental)
% ****************************************************************************************************
% ********************************************************************
% Changing the text area
% ********************************************************************
%\areaset[current]{312pt}{761pt} % 686 (factor 2.2) + 33 head + 42 head \the\footskip
%\setlength{\marginparwidth}{7em}%
%\setlength{\marginparsep}{2em}%

% from the sty file.
% \ifthenelse{\equal{\ct@paper}{letter}}%
%     {% Letter 216mm x 279mm
%             \PackageInfo{classicthesis}{letter paper, Palatino or other}
% \areaset[current]{356pt}{700pt}%  guessing from A4 values, 356*1.75 +  + 33 head + 42 head \the\footskip

% Bringhurst suggests that the Palatino 11pt alphabet length being 145pt, we should use 28 pica = 28*12=336 as the width for 66 characters; but 30 pica will give 70 characters for a 11pt font. 30pica = 30*12=360
% at 12pt palatino, we have 159.67 pt alphabet width, at 30 pica, we have 65 characters per line, at 32pica, we have 69 characters per line.


% in classic thesis, this is what I get
% The footskip is: 50.75pt
% The head height is: 17.0pt
% The head separation is: 21.75pt
%  The sum is 89.5 pt not 75

% use to double check

% The footskip is: \printlength{\footskip}\\
% The head height is: \printlength{\headheight}\\
% The head separation is: \printlength{\headsep}

% this means that the original calculation in the classic thesis sty is incorrect (original calculation is 75=+ 33 head + 42 head \the\footskip)

% % using golden section, 30 pica
% \areaset[current]{360pt}{673pt}%   356*1.62 (golden section) +  89.5 (headep+headheight+footskip)
% \setlength{\marginparwidth}{7em}%
% \setlength{\marginparsep}{2em}%

% using the factor of 1.73 (hexagon), 30 pica, looks good with 11pt.
% \areaset[current]{360pt}{712pt}%   360*1.73 (hexagon) +  89.5 (headep+headheight+footskip)
% \setlength{\marginparwidth}{7em}%
% \setlength{\marginparsep}{2em}%

% % using the factor of 1.73 (hexagon), 32 pica, 12pt only!!
% \areaset[current]{384pt}{754pt}%   384*1.73 (hexagon) +  89.5 (headep+headheight+footskip), new calculation
% \areaset[current]{384pt}{739pt}%   384*1.73 (hexagon) +  75 (headep+headheight+footskip), old calculation
% \setlength{\marginparwidth}{7em}%
% \setlength{\marginparsep}{2em}%

% % using the factor of 1.62 (goldensection), 32 pica, 12pt only!!
\areaset[current]{384pt}{712pt}%   384*1.62 (goldensection) +  89.5 (headep+headheight+footskip), new calculation
% \areaset[current]{384pt}{697pt}%   384*1.62 (hexagon) +  75 (headep+headheight+footskip), old calculation
\setlength{\marginparwidth}{7em}%
\setlength{\marginparsep}{2em}%





% ********************************************************************
% Using different fonts
% ********************************************************************
%\usepackage[oldstylenums]{kpfonts} % oldstyle notextcomp
% \usepackage[libertine]{newtxmath}
% \usepackage[osf]{libertine}

%\usepackage[light,condensed,math]{iwona}
%\renewcommand{\sfdefault}{iwona}
%\usepackage{lmodern} % <-- no osf support :-(
%\usepackage{cfr-lm} %
%\usepackage[urw-garamond]{mathdesign} <-- no osf support :-(
%\usepackage[default,osfigures]{opensans} % scale=0.95
%\usepackage[sfdefault]{FiraSans}
% \usepackage[opticals,mathlf]{MinionPro} % onlytext
% ********************************************************************
% \usepackage[largesc,osf]{newpxtext} % smallcaps is 8% bigger with largesc
\usepackage[theoremfont,tighter,p,osf,largesc]{newpxtext}
\linespread{1.05} % a bit more for Palatino
\usepackage[T1]{fontenc}
% \usepackage{textcomp} % required for special glyphs 
\usepackage[bigdelims,vvarbb]{newpxmath} % too many math alphabets error
% \usepackage[scr=rsfso]{mathalfa}% \mathscr is fancier than \mathcal

\usepackage{sidenotes}


% Used to fix these:
% https://bitbucket.org/amiede/classicthesis/issues/139/italics-in-pallatino-capitals-chapter
% https://bitbucket.org/amiede/classicthesis/issues/45/problema-testatine-su-classicthesis-style
% ********************************************************************
% ****************************************************************************************************

%% microtype
% \usepackage{microtype}
% ****************************************************************************************************
% If you like the classicthesis, then I would appreciate a postcard.
% My address can be found in the file ClassicThesis.pdf. A collection
% of the postcards I received so far is available online at
% http://postcards.miede.de
% ****************************************************************************************************


% ****************************************************************************************************
% 0. Set the encoding of your files. UTF-8 is the only sensible encoding nowadays. If you can't read
% äöüßáéçèê∂åëæƒÏ€ then change the encoding setting in your editor, not the line below. If your editor
% does not support utf8 use another editor!
% ****************************************************************************************************
\PassOptionsToPackage{utf8}{inputenc}
  \usepackage{inputenc}

\PassOptionsToPackage{T1}{fontenc} % T2A for cyrillics
  \usepackage{fontenc}


% ****************************************************************************************************
% 1. Configure classicthesis for your needs here, e.g., remove "drafting" below
% in order to deactivate the time-stamp on the pages
% (see ClassicThesis.pdf for more information):
% ****************************************************************************************************
\PassOptionsToPackage{
  drafting=true,    % print version information on the bottom of the pages
  tocaligned=false, % the left column of the toc will be aligned (no indentation)
  dottedtoc=false,  % page numbers in ToC flushed right
  eulerchapternumbers=true, % use AMS Euler for chapter font (otherwise Palatino)
  linedheaders=false,       % chaper headers will have line above and beneath
  floatperchapter=true,     % numbering per chapter for all floats (i.e., Figure 1.1)
  eulermath=false,  % use awesome Euler fonts for mathematical formulae (only with pdfLaTeX)
  beramono=true,    % toggle a nice monospaced font (w/ bold)
  palatino=false,    % deactivate standard font for loading another one, see the last section at the end of this file for suggestions
  style=classicthesis % classicthesis, arsclassica
}{classicthesis}


% ****************************************************************************************************
% 2. Personal data and user ad-hoc commands (insert your own data here)
% ****************************************************************************************************
\newcommand{\myTitle}{Controllable Natural Langauge Generation for Audience-Centric Styles\xspace}
\newcommand{\mySubtitle}{An Homage to The Elements of Typographic Style\xspace}
\newcommand{\myDegree}{\textsc{Thesis}\\
Submitted in partial fulfillment of the requirements \\
for the degree of Master of Science in Computer Science\\ in the Graduate College of the \\ University of Illinois at Urbana-Champaign, 2023\xspace}
\newcommand{\myName}{Samraj Moorjani\xspace}
\newcommand{\myProf}{Hari Sundaram\xspace}
\newcommand{\myOtherProf}{Put name here\xspace}
\newcommand{\mySupervisor}{Put name here\xspace}
\newcommand{\myFaculty}{\noindent
% \noindent \textsc{Doctoral Committee}:\\
  % \indent Associate Professor Hari Sundaram, Chair, \\
  % \indent Assistant Professor Alexander Schwing, \\
  % \indent Professor ChengXiang Zhai,\\
  % \indent Dr. Chris Brew
  % \xspace
  }
\newcommand{\myDepartment}{Department of Computer Science\xspace}
\newcommand{\myUni}{University of Illinois, Urbana-Champaign\xspace}
\newcommand{\myLocation}{Urbana, Illinois \xspace}
\newcommand{\myTime}{May 2023\xspace}
\newcommand{\myVersion}{\classicthesis}

% ********************************************************************
% Setup, finetuning, and useful commands
% ********************************************************************
\providecommand{\mLyX}{L\kern-.1667em\lower.25em\hbox{Y}\kern-.125emX\@}
\newcommand{\ie}{i.\,e.}
\newcommand{\Ie}{I.\,e.}
\newcommand{\eg}{e.\,g.}
\newcommand{\Eg}{E.\,g.}
% ****************************************************************************************************


% ****************************************************************************************************
% 3. Loading some handy packages
% ****************************************************************************************************
% ********************************************************************
% Packages with options that might require adjustments
% ********************************************************************
\PassOptionsToPackage{ngerman,american}{babel} % change this to your language(s), main language last
% Spanish languages need extra options in order to work with this template
%\PassOptionsToPackage{spanish,es-lcroman}{babel}
    \usepackage{babel}

\DeclareUnicodeCharacter{0301}{HERE}
\usepackage{csquotes}
\PassOptionsToPackage{%
  backend=biber,
  bibencoding=utf8, %instead of bibtex
  % backend=bibtex8,
  % bibencoding=ascii,%
  doi=true,
  url=false,
  language=auto,%
  % style=numeric-comp,%
  style=authoryear-comp, % Author 1999, 2010
  %bibstyle=authoryear,dashed=false, % dashed: substitute rep. author with ---
  sorting=nyt, % name, year, title
  maxbibnames=10, % default: 3, et al.
  backref=true,%
  natbib=true % natbib compatibility mode (\citep and \citet still work)
}{biblatex}
    \usepackage{biblatex}

% avoids the too many math alphabets error
% https://www.texfaq.org/FAQ-manymathalph.html
% \newcommand\hmmax{0} % default is 3, restricts the number math groups; TeX doesn't allow more than 16
\newcommand\bmmax{2} % default is 4

\PassOptionsToPackage{fleqn}{amsmath}       % math environments and more by the AMS
  \usepackage{amsmath}

% ********************************************************************
% General useful packages
% ********************************************************************
\usepackage{graphicx} %
\usepackage{scrhack} % fix warnings when using KOMA with listings package
\usepackage{xspace} % to get the spacing after macros right
\PassOptionsToPackage{printonlyused,smaller}{acronym}
  \usepackage{acronym} % nice macros for handling all acronyms in the thesis
  %\renewcommand{\bflabel}[1]{{#1}\hfill} % fix the list of acronyms --> no longer working
  %\renewcommand*{\acsfont}[1]{\textsc{#1}}
  %\renewcommand*{\aclabelfont}[1]{\acsfont{#1}}
  %\def\bflabel#1{{#1\hfill}}
  \def\bflabel#1{{\acsfont{#1}\hfill}}
  \def\aclabelfont#1{\acsfont{#1}}
% ****************************************************************************************************
%\usepackage{pgfplots} % External TikZ/PGF support (thanks to Andreas Nautsch)
%\usetikzlibrary{external}
%\tikzexternalize[mode=list and make, prefix=ext-tikz/]
% ****************************************************************************************************


% ****************************************************************************************************
% 4. Setup floats: tables, (sub)figures, and captions
% ****************************************************************************************************
\usepackage{tabularx} % better tables
  \setlength{\extrarowheight}{3pt} % increase table row height
\newcommand{\tableheadline}[1]{\multicolumn{1}{l}{\spacedlowsmallcaps{#1}}}
\newcommand{\myfloatalign}{\centering} % to be used with each float for alignment
% \usepackage{side}

\usepackage[counterclockwise]{rotating}

% \usepackage{sidenotes}
% \usepackage{subfig} % clash with subcaption
% ****************************************************************************************************


% ****************************************************************************************************
% 5. Setup code listings
% ****************************************************************************************************
\usepackage{listings}
%\lstset{emph={trueIndex,root},emphstyle=\color{BlueViolet}}%\underbar} % for special keywords
\lstset{language=[LaTeX]Tex,%C++,
  morekeywords={PassOptionsToPackage,selectlanguage},
  keywordstyle=\color{RoyalBlue},%\bfseries,
  basicstyle=\small\ttfamily,
  %identifierstyle=\color{NavyBlue},
  commentstyle=\color{Green}\ttfamily,
  stringstyle=\rmfamily,
  numbers=none,%left,%
  numberstyle=\scriptsize,%\tiny
  stepnumber=5,
  numbersep=8pt,
  showstringspaces=false,
  breaklines=true,
  %frameround=ftff,
  %frame=single,
  belowcaptionskip=.75\baselineskip
  %frame=L
}
% ****************************************************************************************************

% Yes, classicthesis loads microtype with the option pdfspacing. You can change the options by issuing

\PassOptionsToPackage{kerning=true, tracking=false}{microtype}


% ****************************************************************************************************
% 6. Last calls before the bar closes
% ****************************************************************************************************
% ********************************************************************
% Her Majesty herself
% ********************************************************************
\usepackage{classicthesis}


% ********************************************************************
% Fine-tune hyperreferences (hyperref should be called last)
% ********************************************************************
\hypersetup{%
  %draft, % hyperref's draft mode, for printing see below
  colorlinks=true, linktocpage=true, pdfstartpage=3, pdfstartview=FitV,%
  % uncomment the following line if you want to have black links (e.g., for printing)
  %colorlinks=false, linktocpage=false, pdfstartpage=3, pdfstartview=FitV, pdfborder={0 0 0},%
  breaklinks=true, pageanchor=true,%
  pdfpagemode=UseNone, %
  % pdfpagemode=UseOutlines,%
  plainpages=false, bookmarksnumbered, bookmarksopen=true, bookmarksopenlevel=1,%
  hypertexnames=true, pdfhighlight=/O,%nesting=true,%frenchlinks,%
  urlcolor=CTurl, linkcolor=CTlink, citecolor=CTcitation, %pagecolor=RoyalBlue,%
  %urlcolor=Black, linkcolor=Black, citecolor=Black, %pagecolor=Black,%
  pdftitle={\myTitle},%
  pdfauthor={\textcopyright\ \myName, \myUni, \myFaculty},%
  pdfsubject={},%
  pdfkeywords={},%
  pdfcreator={pdfLaTeX},%
  pdfproducer={LaTeX with hyperref and classicthesis}%
}


% ********************************************************************
% Setup autoreferences (hyperref and babel)
% ********************************************************************
% There are some issues regarding autorefnames
% http://www.tex.ac.uk/cgi-bin/texfaq2html?label=latexwords
% you have to redefine the macros for the
% language you use, e.g., american, ngerman
% (as chosen when loading babel/AtBeginDocument)
% ********************************************************************
\makeatletter
\@ifpackageloaded{babel}%
  {%
    \addto\extrasamerican{%
      \renewcommand*{\figureautorefname}{Figure}%
      \renewcommand*{\tableautorefname}{Table}%
      \renewcommand*{\partautorefname}{Part}%
      \renewcommand*{\chapterautorefname}{Chapter}%
      \renewcommand*{\sectionautorefname}{Section}%
      \renewcommand*{\subsectionautorefname}{Section}%
      \renewcommand*{\subsubsectionautorefname}{Section}%
    }%
    \addto\extrasngerman{%
      \renewcommand*{\paragraphautorefname}{Absatz}%
      \renewcommand*{\subparagraphautorefname}{Unterabsatz}%
      \renewcommand*{\footnoteautorefname}{Fu\"snote}%
      \renewcommand*{\FancyVerbLineautorefname}{Zeile}%
      \renewcommand*{\theoremautorefname}{Theorem}%
      \renewcommand*{\appendixautorefname}{Anhang}%
      \renewcommand*{\equationautorefname}{Gleichung}%
      \renewcommand*{\itemautorefname}{Punkt}%
    }%
      % Fix to getting autorefs for subfigures right (thanks to Belinda Vogt for changing the definition)
      \providecommand{\subfigureautorefname}{\figureautorefname}%
    }{\relax}
\makeatother


% ********************************************************************
% Development Stuff
% ********************************************************************
\listfiles
%\PassOptionsToPackage{l2tabu,orthodox,abort}{nag}
%  \usepackage{nag}
%\PassOptionsToPackage{warning, all}{onlyamsmath}
%  \usepackage{onlyamsmath}


% ****************************************************************************************************
% 7. Further adjustments (experimental)
% ****************************************************************************************************
% ********************************************************************
% Changing the text area
% ********************************************************************
%\areaset[current]{312pt}{761pt} % 686 (factor 2.2) + 33 head + 42 head \the\footskip
%\setlength{\marginparwidth}{7em}%
%\setlength{\marginparsep}{2em}%

% from the sty file.
% \ifthenelse{\equal{\ct@paper}{letter}}%
%     {% Letter 216mm x 279mm
%             \PackageInfo{classicthesis}{letter paper, Palatino or other}
% \areaset[current]{356pt}{700pt}%  guessing from A4 values, 356*1.75 +  + 33 head + 42 head \the\footskip

% Bringhurst suggests that the Palatino 11pt alphabet length being 145pt, we should use 28 pica = 28*12=336 as the width for 66 characters; but 30 pica will give 70 characters for a 11pt font. 30pica = 30*12=360
% at 12pt palatino, we have 159.67 pt alphabet width, at 30 pica, we have 65 characters per line, at 32pica, we have 69 characters per line.


% in classic thesis, this is what I get
% The footskip is: 50.75pt
% The head height is: 17.0pt
% The head separation is: 21.75pt
%  The sum is 89.5 pt not 75

% use to double check

% The footskip is: \printlength{\footskip}\\
% The head height is: \printlength{\headheight}\\
% The head separation is: \printlength{\headsep}

% this means that the original calculation in the classic thesis sty is incorrect (original calculation is 75=+ 33 head + 42 head \the\footskip)

% % using golden section, 30 pica
% \areaset[current]{360pt}{673pt}%   356*1.62 (golden section) +  89.5 (headep+headheight+footskip)
% \setlength{\marginparwidth}{7em}%
% \setlength{\marginparsep}{2em}%

% using the factor of 1.73 (hexagon), 30 pica, looks good with 11pt.
% \areaset[current]{360pt}{712pt}%   360*1.73 (hexagon) +  89.5 (headep+headheight+footskip)
% \setlength{\marginparwidth}{7em}%
% \setlength{\marginparsep}{2em}%

% % using the factor of 1.73 (hexagon), 32 pica, 12pt only!!
% \areaset[current]{384pt}{754pt}%   384*1.73 (hexagon) +  89.5 (headep+headheight+footskip), new calculation
% \areaset[current]{384pt}{739pt}%   384*1.73 (hexagon) +  75 (headep+headheight+footskip), old calculation
% \setlength{\marginparwidth}{7em}%
% \setlength{\marginparsep}{2em}%

% % using the factor of 1.62 (goldensection), 32 pica, 12pt only!!
\areaset[current]{384pt}{712pt}%   384*1.62 (goldensection) +  89.5 (headep+headheight+footskip), new calculation
% \areaset[current]{384pt}{697pt}%   384*1.62 (hexagon) +  75 (headep+headheight+footskip), old calculation
\setlength{\marginparwidth}{7em}%
\setlength{\marginparsep}{2em}%





% ********************************************************************
% Using different fonts
% ********************************************************************
%\usepackage[oldstylenums]{kpfonts} % oldstyle notextcomp
% \usepackage[libertine]{newtxmath}
% \usepackage[osf]{libertine}

%\usepackage[light,condensed,math]{iwona}
%\renewcommand{\sfdefault}{iwona}
%\usepackage{lmodern} % <-- no osf support :-(
%\usepackage{cfr-lm} %
%\usepackage[urw-garamond]{mathdesign} <-- no osf support :-(
%\usepackage[default,osfigures]{opensans} % scale=0.95
%\usepackage[sfdefault]{FiraSans}
% \usepackage[opticals,mathlf]{MinionPro} % onlytext
% ********************************************************************
% \usepackage[largesc,osf]{newpxtext} % smallcaps is 8% bigger with largesc
\usepackage[theoremfont,tighter,p,osf,largesc]{newpxtext}
\linespread{1.05} % a bit more for Palatino
\usepackage[T1]{fontenc}
% \usepackage{textcomp} % required for special glyphs 
\usepackage[bigdelims,vvarbb]{newpxmath} % too many math alphabets error
% \usepackage[scr=rsfso]{mathalfa}% \mathscr is fancier than \mathcal

\usepackage{sidenotes}


% Used to fix these:
% https://bitbucket.org/amiede/classicthesis/issues/139/italics-in-pallatino-capitals-chapter
% https://bitbucket.org/amiede/classicthesis/issues/45/problema-testatine-su-classicthesis-style
% ********************************************************************
% ****************************************************************************************************

%% microtype
% \usepackage{microtype}
% ****************************************************************************************************
% If you like the classicthesis, then I would appreciate a postcard.
% My address can be found in the file ClassicThesis.pdf. A collection
% of the postcards I received so far is available online at
% http://postcards.miede.de
% ****************************************************************************************************


% ****************************************************************************************************
% 0. Set the encoding of your files. UTF-8 is the only sensible encoding nowadays. If you can't read
% äöüßáéçèê∂åëæƒÏ€ then change the encoding setting in your editor, not the line below. If your editor
% does not support utf8 use another editor!
% ****************************************************************************************************
\PassOptionsToPackage{utf8}{inputenc}
  \usepackage{inputenc}

\PassOptionsToPackage{T1}{fontenc} % T2A for cyrillics
  \usepackage{fontenc}


% ****************************************************************************************************
% 1. Configure classicthesis for your needs here, e.g., remove "drafting" below
% in order to deactivate the time-stamp on the pages
% (see ClassicThesis.pdf for more information):
% ****************************************************************************************************
\PassOptionsToPackage{
  drafting=true,    % print version information on the bottom of the pages
  tocaligned=false, % the left column of the toc will be aligned (no indentation)
  dottedtoc=false,  % page numbers in ToC flushed right
  eulerchapternumbers=true, % use AMS Euler for chapter font (otherwise Palatino)
  linedheaders=false,       % chaper headers will have line above and beneath
  floatperchapter=true,     % numbering per chapter for all floats (i.e., Figure 1.1)
  eulermath=false,  % use awesome Euler fonts for mathematical formulae (only with pdfLaTeX)
  beramono=true,    % toggle a nice monospaced font (w/ bold)
  palatino=false,    % deactivate standard font for loading another one, see the last section at the end of this file for suggestions
  style=classicthesis % classicthesis, arsclassica
}{classicthesis}


% ****************************************************************************************************
% 2. Personal data and user ad-hoc commands (insert your own data here)
% ****************************************************************************************************
\newcommand{\myTitle}{Controllable Natural Langauge Generation for Audience-Centric Styles\xspace}
\newcommand{\mySubtitle}{An Homage to The Elements of Typographic Style\xspace}
\newcommand{\myDegree}{\textsc{Thesis}\\
Submitted in partial fulfillment of the requirements \\
for the degree of Master of Science in Computer Science\\ in the Graduate College of the \\ University of Illinois at Urbana-Champaign, 2023\xspace}
\newcommand{\myName}{Samraj Moorjani\xspace}
\newcommand{\myProf}{Hari Sundaram\xspace}
\newcommand{\myOtherProf}{Put name here\xspace}
\newcommand{\mySupervisor}{Put name here\xspace}
\newcommand{\myFaculty}{\noindent
% \noindent \textsc{Doctoral Committee}:\\
  % \indent Associate Professor Hari Sundaram, Chair, \\
  % \indent Assistant Professor Alexander Schwing, \\
  % \indent Professor ChengXiang Zhai,\\
  % \indent Dr. Chris Brew
  % \xspace
  }
\newcommand{\myDepartment}{Department of Computer Science\xspace}
\newcommand{\myUni}{University of Illinois, Urbana-Champaign\xspace}
\newcommand{\myLocation}{Urbana, Illinois \xspace}
\newcommand{\myTime}{May 2023\xspace}
\newcommand{\myVersion}{\classicthesis}

% ********************************************************************
% Setup, finetuning, and useful commands
% ********************************************************************
\providecommand{\mLyX}{L\kern-.1667em\lower.25em\hbox{Y}\kern-.125emX\@}
\newcommand{\ie}{i.\,e.}
\newcommand{\Ie}{I.\,e.}
\newcommand{\eg}{e.\,g.}
\newcommand{\Eg}{E.\,g.}
% ****************************************************************************************************


% ****************************************************************************************************
% 3. Loading some handy packages
% ****************************************************************************************************
% ********************************************************************
% Packages with options that might require adjustments
% ********************************************************************
\PassOptionsToPackage{ngerman,american}{babel} % change this to your language(s), main language last
% Spanish languages need extra options in order to work with this template
%\PassOptionsToPackage{spanish,es-lcroman}{babel}
    \usepackage{babel}

\DeclareUnicodeCharacter{0301}{HERE}
\usepackage{csquotes}
\PassOptionsToPackage{%
  backend=biber,
  bibencoding=utf8, %instead of bibtex
  % backend=bibtex8,
  % bibencoding=ascii,%
  doi=true,
  url=false,
  language=auto,%
  % style=numeric-comp,%
  style=authoryear-comp, % Author 1999, 2010
  %bibstyle=authoryear,dashed=false, % dashed: substitute rep. author with ---
  sorting=nyt, % name, year, title
  maxbibnames=10, % default: 3, et al.
  backref=true,%
  natbib=true % natbib compatibility mode (\citep and \citet still work)
}{biblatex}
    \usepackage{biblatex}

% avoids the too many math alphabets error
% https://www.texfaq.org/FAQ-manymathalph.html
% \newcommand\hmmax{0} % default is 3, restricts the number math groups; TeX doesn't allow more than 16
\newcommand\bmmax{2} % default is 4

\PassOptionsToPackage{fleqn}{amsmath}       % math environments and more by the AMS
  \usepackage{amsmath}

% ********************************************************************
% General useful packages
% ********************************************************************
\usepackage{graphicx} %
\usepackage{scrhack} % fix warnings when using KOMA with listings package
\usepackage{xspace} % to get the spacing after macros right
\PassOptionsToPackage{printonlyused,smaller}{acronym}
  \usepackage{acronym} % nice macros for handling all acronyms in the thesis
  %\renewcommand{\bflabel}[1]{{#1}\hfill} % fix the list of acronyms --> no longer working
  %\renewcommand*{\acsfont}[1]{\textsc{#1}}
  %\renewcommand*{\aclabelfont}[1]{\acsfont{#1}}
  %\def\bflabel#1{{#1\hfill}}
  \def\bflabel#1{{\acsfont{#1}\hfill}}
  \def\aclabelfont#1{\acsfont{#1}}
% ****************************************************************************************************
%\usepackage{pgfplots} % External TikZ/PGF support (thanks to Andreas Nautsch)
%\usetikzlibrary{external}
%\tikzexternalize[mode=list and make, prefix=ext-tikz/]
% ****************************************************************************************************


% ****************************************************************************************************
% 4. Setup floats: tables, (sub)figures, and captions
% ****************************************************************************************************
\usepackage{tabularx} % better tables
  \setlength{\extrarowheight}{3pt} % increase table row height
\newcommand{\tableheadline}[1]{\multicolumn{1}{l}{\spacedlowsmallcaps{#1}}}
\newcommand{\myfloatalign}{\centering} % to be used with each float for alignment
% \usepackage{side}

\usepackage[counterclockwise]{rotating}

% \usepackage{sidenotes}
% \usepackage{subfig} % clash with subcaption
% ****************************************************************************************************


% ****************************************************************************************************
% 5. Setup code listings
% ****************************************************************************************************
\usepackage{listings}
%\lstset{emph={trueIndex,root},emphstyle=\color{BlueViolet}}%\underbar} % for special keywords
\lstset{language=[LaTeX]Tex,%C++,
  morekeywords={PassOptionsToPackage,selectlanguage},
  keywordstyle=\color{RoyalBlue},%\bfseries,
  basicstyle=\small\ttfamily,
  %identifierstyle=\color{NavyBlue},
  commentstyle=\color{Green}\ttfamily,
  stringstyle=\rmfamily,
  numbers=none,%left,%
  numberstyle=\scriptsize,%\tiny
  stepnumber=5,
  numbersep=8pt,
  showstringspaces=false,
  breaklines=true,
  %frameround=ftff,
  %frame=single,
  belowcaptionskip=.75\baselineskip
  %frame=L
}
% ****************************************************************************************************

% Yes, classicthesis loads microtype with the option pdfspacing. You can change the options by issuing

\PassOptionsToPackage{kerning=true, tracking=false}{microtype}


% ****************************************************************************************************
% 6. Last calls before the bar closes
% ****************************************************************************************************
% ********************************************************************
% Her Majesty herself
% ********************************************************************
\usepackage{classicthesis}


% ********************************************************************
% Fine-tune hyperreferences (hyperref should be called last)
% ********************************************************************
\hypersetup{%
  %draft, % hyperref's draft mode, for printing see below
  colorlinks=true, linktocpage=true, pdfstartpage=3, pdfstartview=FitV,%
  % uncomment the following line if you want to have black links (e.g., for printing)
  %colorlinks=false, linktocpage=false, pdfstartpage=3, pdfstartview=FitV, pdfborder={0 0 0},%
  breaklinks=true, pageanchor=true,%
  pdfpagemode=UseNone, %
  % pdfpagemode=UseOutlines,%
  plainpages=false, bookmarksnumbered, bookmarksopen=true, bookmarksopenlevel=1,%
  hypertexnames=true, pdfhighlight=/O,%nesting=true,%frenchlinks,%
  urlcolor=CTurl, linkcolor=CTlink, citecolor=CTcitation, %pagecolor=RoyalBlue,%
  %urlcolor=Black, linkcolor=Black, citecolor=Black, %pagecolor=Black,%
  pdftitle={\myTitle},%
  pdfauthor={\textcopyright\ \myName, \myUni, \myFaculty},%
  pdfsubject={},%
  pdfkeywords={},%
  pdfcreator={pdfLaTeX},%
  pdfproducer={LaTeX with hyperref and classicthesis}%
}


% ********************************************************************
% Setup autoreferences (hyperref and babel)
% ********************************************************************
% There are some issues regarding autorefnames
% http://www.tex.ac.uk/cgi-bin/texfaq2html?label=latexwords
% you have to redefine the macros for the
% language you use, e.g., american, ngerman
% (as chosen when loading babel/AtBeginDocument)
% ********************************************************************
\makeatletter
\@ifpackageloaded{babel}%
  {%
    \addto\extrasamerican{%
      \renewcommand*{\figureautorefname}{Figure}%
      \renewcommand*{\tableautorefname}{Table}%
      \renewcommand*{\partautorefname}{Part}%
      \renewcommand*{\chapterautorefname}{Chapter}%
      \renewcommand*{\sectionautorefname}{Section}%
      \renewcommand*{\subsectionautorefname}{Section}%
      \renewcommand*{\subsubsectionautorefname}{Section}%
    }%
    \addto\extrasngerman{%
      \renewcommand*{\paragraphautorefname}{Absatz}%
      \renewcommand*{\subparagraphautorefname}{Unterabsatz}%
      \renewcommand*{\footnoteautorefname}{Fu\"snote}%
      \renewcommand*{\FancyVerbLineautorefname}{Zeile}%
      \renewcommand*{\theoremautorefname}{Theorem}%
      \renewcommand*{\appendixautorefname}{Anhang}%
      \renewcommand*{\equationautorefname}{Gleichung}%
      \renewcommand*{\itemautorefname}{Punkt}%
    }%
      % Fix to getting autorefs for subfigures right (thanks to Belinda Vogt for changing the definition)
      \providecommand{\subfigureautorefname}{\figureautorefname}%
    }{\relax}
\makeatother


% ********************************************************************
% Development Stuff
% ********************************************************************
\listfiles
%\PassOptionsToPackage{l2tabu,orthodox,abort}{nag}
%  \usepackage{nag}
%\PassOptionsToPackage{warning, all}{onlyamsmath}
%  \usepackage{onlyamsmath}


% ****************************************************************************************************
% 7. Further adjustments (experimental)
% ****************************************************************************************************
% ********************************************************************
% Changing the text area
% ********************************************************************
%\areaset[current]{312pt}{761pt} % 686 (factor 2.2) + 33 head + 42 head \the\footskip
%\setlength{\marginparwidth}{7em}%
%\setlength{\marginparsep}{2em}%

% from the sty file.
% \ifthenelse{\equal{\ct@paper}{letter}}%
%     {% Letter 216mm x 279mm
%             \PackageInfo{classicthesis}{letter paper, Palatino or other}
% \areaset[current]{356pt}{700pt}%  guessing from A4 values, 356*1.75 +  + 33 head + 42 head \the\footskip

% Bringhurst suggests that the Palatino 11pt alphabet length being 145pt, we should use 28 pica = 28*12=336 as the width for 66 characters; but 30 pica will give 70 characters for a 11pt font. 30pica = 30*12=360
% at 12pt palatino, we have 159.67 pt alphabet width, at 30 pica, we have 65 characters per line, at 32pica, we have 69 characters per line.


% in classic thesis, this is what I get
% The footskip is: 50.75pt
% The head height is: 17.0pt
% The head separation is: 21.75pt
%  The sum is 89.5 pt not 75

% use to double check

% The footskip is: \printlength{\footskip}\\
% The head height is: \printlength{\headheight}\\
% The head separation is: \printlength{\headsep}

% this means that the original calculation in the classic thesis sty is incorrect (original calculation is 75=+ 33 head + 42 head \the\footskip)

% % using golden section, 30 pica
% \areaset[current]{360pt}{673pt}%   356*1.62 (golden section) +  89.5 (headep+headheight+footskip)
% \setlength{\marginparwidth}{7em}%
% \setlength{\marginparsep}{2em}%

% using the factor of 1.73 (hexagon), 30 pica, looks good with 11pt.
% \areaset[current]{360pt}{712pt}%   360*1.73 (hexagon) +  89.5 (headep+headheight+footskip)
% \setlength{\marginparwidth}{7em}%
% \setlength{\marginparsep}{2em}%

% % using the factor of 1.73 (hexagon), 32 pica, 12pt only!!
% \areaset[current]{384pt}{754pt}%   384*1.73 (hexagon) +  89.5 (headep+headheight+footskip), new calculation
% \areaset[current]{384pt}{739pt}%   384*1.73 (hexagon) +  75 (headep+headheight+footskip), old calculation
% \setlength{\marginparwidth}{7em}%
% \setlength{\marginparsep}{2em}%

% % using the factor of 1.62 (goldensection), 32 pica, 12pt only!!
\areaset[current]{384pt}{712pt}%   384*1.62 (goldensection) +  89.5 (headep+headheight+footskip), new calculation
% \areaset[current]{384pt}{697pt}%   384*1.62 (hexagon) +  75 (headep+headheight+footskip), old calculation
\setlength{\marginparwidth}{7em}%
\setlength{\marginparsep}{2em}%





% ********************************************************************
% Using different fonts
% ********************************************************************
%\usepackage[oldstylenums]{kpfonts} % oldstyle notextcomp
% \usepackage[libertine]{newtxmath}
% \usepackage[osf]{libertine}

%\usepackage[light,condensed,math]{iwona}
%\renewcommand{\sfdefault}{iwona}
%\usepackage{lmodern} % <-- no osf support :-(
%\usepackage{cfr-lm} %
%\usepackage[urw-garamond]{mathdesign} <-- no osf support :-(
%\usepackage[default,osfigures]{opensans} % scale=0.95
%\usepackage[sfdefault]{FiraSans}
% \usepackage[opticals,mathlf]{MinionPro} % onlytext
% ********************************************************************
% \usepackage[largesc,osf]{newpxtext} % smallcaps is 8% bigger with largesc
\usepackage[theoremfont,tighter,p,osf,largesc]{newpxtext}
\linespread{1.05} % a bit more for Palatino
\usepackage[T1]{fontenc}
% \usepackage{textcomp} % required for special glyphs 
\usepackage[bigdelims,vvarbb]{newpxmath} % too many math alphabets error
% \usepackage[scr=rsfso]{mathalfa}% \mathscr is fancier than \mathcal

\usepackage{sidenotes}


% Used to fix these:
% https://bitbucket.org/amiede/classicthesis/issues/139/italics-in-pallatino-capitals-chapter
% https://bitbucket.org/amiede/classicthesis/issues/45/problema-testatine-su-classicthesis-style
% ********************************************************************
% ****************************************************************************************************

%% microtype
% \usepackage{microtype}
% ****************************************************************************************************
% If you like the classicthesis, then I would appreciate a postcard.
% My address can be found in the file ClassicThesis.pdf. A collection
% of the postcards I received so far is available online at
% http://postcards.miede.de
% ****************************************************************************************************


% ****************************************************************************************************
% 0. Set the encoding of your files. UTF-8 is the only sensible encoding nowadays. If you can't read
% äöüßáéçèê∂åëæƒÏ€ then change the encoding setting in your editor, not the line below. If your editor
% does not support utf8 use another editor!
% ****************************************************************************************************
\PassOptionsToPackage{utf8}{inputenc}
  \usepackage{inputenc}

\PassOptionsToPackage{T1}{fontenc} % T2A for cyrillics
  \usepackage{fontenc}


% ****************************************************************************************************
% 1. Configure classicthesis for your needs here, e.g., remove "drafting" below
% in order to deactivate the time-stamp on the pages
% (see ClassicThesis.pdf for more information):
% ****************************************************************************************************
\PassOptionsToPackage{
  drafting=true,    % print version information on the bottom of the pages
  tocaligned=false, % the left column of the toc will be aligned (no indentation)
  dottedtoc=false,  % page numbers in ToC flushed right
  eulerchapternumbers=true, % use AMS Euler for chapter font (otherwise Palatino)
  linedheaders=false,       % chaper headers will have line above and beneath
  floatperchapter=true,     % numbering per chapter for all floats (i.e., Figure 1.1)
  eulermath=false,  % use awesome Euler fonts for mathematical formulae (only with pdfLaTeX)
  beramono=true,    % toggle a nice monospaced font (w/ bold)
  palatino=false,    % deactivate standard font for loading another one, see the last section at the end of this file for suggestions
  style=classicthesis % classicthesis, arsclassica
}{classicthesis}


% ****************************************************************************************************
% 2. Personal data and user ad-hoc commands (insert your own data here)
% ****************************************************************************************************
\newcommand{\myTitle}{Controllable Natural Langauge Generation for Audience-Centric Styles\xspace}
\newcommand{\mySubtitle}{An Homage to The Elements of Typographic Style\xspace}
\newcommand{\myDegree}{\textsc{Thesis}\\
Submitted in partial fulfillment of the requirements \\
for the degree of Master of Science in Computer Science\\ in the Graduate College of the \\ University of Illinois at Urbana-Champaign, 2023\xspace}
\newcommand{\myName}{Samraj Moorjani\xspace}
\newcommand{\myProf}{Hari Sundaram\xspace}
\newcommand{\myOtherProf}{Put name here\xspace}
\newcommand{\mySupervisor}{Put name here\xspace}
\newcommand{\myFaculty}{\noindent
% \noindent \textsc{Doctoral Committee}:\\
  % \indent Associate Professor Hari Sundaram, Chair, \\
  % \indent Assistant Professor Alexander Schwing, \\
  % \indent Professor ChengXiang Zhai,\\
  % \indent Dr. Chris Brew
  % \xspace
  }
\newcommand{\myDepartment}{Department of Computer Science\xspace}
\newcommand{\myUni}{University of Illinois, Urbana-Champaign\xspace}
\newcommand{\myLocation}{Urbana, Illinois \xspace}
\newcommand{\myTime}{May 2023\xspace}
\newcommand{\myVersion}{\classicthesis}

% ********************************************************************
% Setup, finetuning, and useful commands
% ********************************************************************
\providecommand{\mLyX}{L\kern-.1667em\lower.25em\hbox{Y}\kern-.125emX\@}
\newcommand{\ie}{i.\,e.}
\newcommand{\Ie}{I.\,e.}
\newcommand{\eg}{e.\,g.}
\newcommand{\Eg}{E.\,g.}
% ****************************************************************************************************


% ****************************************************************************************************
% 3. Loading some handy packages
% ****************************************************************************************************
% ********************************************************************
% Packages with options that might require adjustments
% ********************************************************************
\PassOptionsToPackage{ngerman,american}{babel} % change this to your language(s), main language last
% Spanish languages need extra options in order to work with this template
%\PassOptionsToPackage{spanish,es-lcroman}{babel}
    \usepackage{babel}

\DeclareUnicodeCharacter{0301}{HERE}
\usepackage{csquotes}
\PassOptionsToPackage{%
  backend=biber,
  bibencoding=utf8, %instead of bibtex
  % backend=bibtex8,
  % bibencoding=ascii,%
  doi=true,
  url=false,
  language=auto,%
  % style=numeric-comp,%
  style=authoryear-comp, % Author 1999, 2010
  %bibstyle=authoryear,dashed=false, % dashed: substitute rep. author with ---
  sorting=nyt, % name, year, title
  maxbibnames=10, % default: 3, et al.
  backref=true,%
  natbib=true % natbib compatibility mode (\citep and \citet still work)
}{biblatex}
    \usepackage{biblatex}

% avoids the too many math alphabets error
% https://www.texfaq.org/FAQ-manymathalph.html
% \newcommand\hmmax{0} % default is 3, restricts the number math groups; TeX doesn't allow more than 16
\newcommand\bmmax{2} % default is 4

\PassOptionsToPackage{fleqn}{amsmath}       % math environments and more by the AMS
  \usepackage{amsmath}

% ********************************************************************
% General useful packages
% ********************************************************************
\usepackage{graphicx} %
\usepackage{scrhack} % fix warnings when using KOMA with listings package
\usepackage{xspace} % to get the spacing after macros right
\PassOptionsToPackage{printonlyused,smaller}{acronym}
  \usepackage{acronym} % nice macros for handling all acronyms in the thesis
  %\renewcommand{\bflabel}[1]{{#1}\hfill} % fix the list of acronyms --> no longer working
  %\renewcommand*{\acsfont}[1]{\textsc{#1}}
  %\renewcommand*{\aclabelfont}[1]{\acsfont{#1}}
  %\def\bflabel#1{{#1\hfill}}
  \def\bflabel#1{{\acsfont{#1}\hfill}}
  \def\aclabelfont#1{\acsfont{#1}}
% ****************************************************************************************************
%\usepackage{pgfplots} % External TikZ/PGF support (thanks to Andreas Nautsch)
%\usetikzlibrary{external}
%\tikzexternalize[mode=list and make, prefix=ext-tikz/]
% ****************************************************************************************************


% ****************************************************************************************************
% 4. Setup floats: tables, (sub)figures, and captions
% ****************************************************************************************************
\usepackage{tabularx} % better tables
  \setlength{\extrarowheight}{3pt} % increase table row height
\newcommand{\tableheadline}[1]{\multicolumn{1}{l}{\spacedlowsmallcaps{#1}}}
\newcommand{\myfloatalign}{\centering} % to be used with each float for alignment
% \usepackage{side}

\usepackage[counterclockwise]{rotating}

% \usepackage{sidenotes}
% \usepackage{subfig} % clash with subcaption
% ****************************************************************************************************


% ****************************************************************************************************
% 5. Setup code listings
% ****************************************************************************************************
\usepackage{listings}
%\lstset{emph={trueIndex,root},emphstyle=\color{BlueViolet}}%\underbar} % for special keywords
\lstset{language=[LaTeX]Tex,%C++,
  morekeywords={PassOptionsToPackage,selectlanguage},
  keywordstyle=\color{RoyalBlue},%\bfseries,
  basicstyle=\small\ttfamily,
  %identifierstyle=\color{NavyBlue},
  commentstyle=\color{Green}\ttfamily,
  stringstyle=\rmfamily,
  numbers=none,%left,%
  numberstyle=\scriptsize,%\tiny
  stepnumber=5,
  numbersep=8pt,
  showstringspaces=false,
  breaklines=true,
  %frameround=ftff,
  %frame=single,
  belowcaptionskip=.75\baselineskip
  %frame=L
}
% ****************************************************************************************************

% Yes, classicthesis loads microtype with the option pdfspacing. You can change the options by issuing

\PassOptionsToPackage{kerning=true, tracking=false}{microtype}


% ****************************************************************************************************
% 6. Last calls before the bar closes
% ****************************************************************************************************
% ********************************************************************
% Her Majesty herself
% ********************************************************************
\usepackage{classicthesis}


% ********************************************************************
% Fine-tune hyperreferences (hyperref should be called last)
% ********************************************************************
\hypersetup{%
  %draft, % hyperref's draft mode, for printing see below
  colorlinks=true, linktocpage=true, pdfstartpage=3, pdfstartview=FitV,%
  % uncomment the following line if you want to have black links (e.g., for printing)
  %colorlinks=false, linktocpage=false, pdfstartpage=3, pdfstartview=FitV, pdfborder={0 0 0},%
  breaklinks=true, pageanchor=true,%
  pdfpagemode=UseNone, %
  % pdfpagemode=UseOutlines,%
  plainpages=false, bookmarksnumbered, bookmarksopen=true, bookmarksopenlevel=1,%
  hypertexnames=true, pdfhighlight=/O,%nesting=true,%frenchlinks,%
  urlcolor=CTurl, linkcolor=CTlink, citecolor=CTcitation, %pagecolor=RoyalBlue,%
  %urlcolor=Black, linkcolor=Black, citecolor=Black, %pagecolor=Black,%
  pdftitle={\myTitle},%
  pdfauthor={\textcopyright\ \myName, \myUni, \myFaculty},%
  pdfsubject={},%
  pdfkeywords={},%
  pdfcreator={pdfLaTeX},%
  pdfproducer={LaTeX with hyperref and classicthesis}%
}


% ********************************************************************
% Setup autoreferences (hyperref and babel)
% ********************************************************************
% There are some issues regarding autorefnames
% http://www.tex.ac.uk/cgi-bin/texfaq2html?label=latexwords
% you have to redefine the macros for the
% language you use, e.g., american, ngerman
% (as chosen when loading babel/AtBeginDocument)
% ********************************************************************
\makeatletter
\@ifpackageloaded{babel}%
  {%
    \addto\extrasamerican{%
      \renewcommand*{\figureautorefname}{Figure}%
      \renewcommand*{\tableautorefname}{Table}%
      \renewcommand*{\partautorefname}{Part}%
      \renewcommand*{\chapterautorefname}{Chapter}%
      \renewcommand*{\sectionautorefname}{Section}%
      \renewcommand*{\subsectionautorefname}{Section}%
      \renewcommand*{\subsubsectionautorefname}{Section}%
    }%
    \addto\extrasngerman{%
      \renewcommand*{\paragraphautorefname}{Absatz}%
      \renewcommand*{\subparagraphautorefname}{Unterabsatz}%
      \renewcommand*{\footnoteautorefname}{Fu\"snote}%
      \renewcommand*{\FancyVerbLineautorefname}{Zeile}%
      \renewcommand*{\theoremautorefname}{Theorem}%
      \renewcommand*{\appendixautorefname}{Anhang}%
      \renewcommand*{\equationautorefname}{Gleichung}%
      \renewcommand*{\itemautorefname}{Punkt}%
    }%
      % Fix to getting autorefs for subfigures right (thanks to Belinda Vogt for changing the definition)
      \providecommand{\subfigureautorefname}{\figureautorefname}%
    }{\relax}
\makeatother


% ********************************************************************
% Development Stuff
% ********************************************************************
\listfiles
%\PassOptionsToPackage{l2tabu,orthodox,abort}{nag}
%  \usepackage{nag}
%\PassOptionsToPackage{warning, all}{onlyamsmath}
%  \usepackage{onlyamsmath}


% ****************************************************************************************************
% 7. Further adjustments (experimental)
% ****************************************************************************************************
% ********************************************************************
% Changing the text area
% ********************************************************************
%\areaset[current]{312pt}{761pt} % 686 (factor 2.2) + 33 head + 42 head \the\footskip
%\setlength{\marginparwidth}{7em}%
%\setlength{\marginparsep}{2em}%

% from the sty file.
% \ifthenelse{\equal{\ct@paper}{letter}}%
%     {% Letter 216mm x 279mm
%             \PackageInfo{classicthesis}{letter paper, Palatino or other}
% \areaset[current]{356pt}{700pt}%  guessing from A4 values, 356*1.75 +  + 33 head + 42 head \the\footskip

% Bringhurst suggests that the Palatino 11pt alphabet length being 145pt, we should use 28 pica = 28*12=336 as the width for 66 characters; but 30 pica will give 70 characters for a 11pt font. 30pica = 30*12=360
% at 12pt palatino, we have 159.67 pt alphabet width, at 30 pica, we have 65 characters per line, at 32pica, we have 69 characters per line.


% in classic thesis, this is what I get
% The footskip is: 50.75pt
% The head height is: 17.0pt
% The head separation is: 21.75pt
%  The sum is 89.5 pt not 75

% use to double check

% The footskip is: \printlength{\footskip}\\
% The head height is: \printlength{\headheight}\\
% The head separation is: \printlength{\headsep}

% this means that the original calculation in the classic thesis sty is incorrect (original calculation is 75=+ 33 head + 42 head \the\footskip)

% % using golden section, 30 pica
% \areaset[current]{360pt}{673pt}%   356*1.62 (golden section) +  89.5 (headep+headheight+footskip)
% \setlength{\marginparwidth}{7em}%
% \setlength{\marginparsep}{2em}%

% using the factor of 1.73 (hexagon), 30 pica, looks good with 11pt.
% \areaset[current]{360pt}{712pt}%   360*1.73 (hexagon) +  89.5 (headep+headheight+footskip)
% \setlength{\marginparwidth}{7em}%
% \setlength{\marginparsep}{2em}%

% % using the factor of 1.73 (hexagon), 32 pica, 12pt only!!
% \areaset[current]{384pt}{754pt}%   384*1.73 (hexagon) +  89.5 (headep+headheight+footskip), new calculation
% \areaset[current]{384pt}{739pt}%   384*1.73 (hexagon) +  75 (headep+headheight+footskip), old calculation
% \setlength{\marginparwidth}{7em}%
% \setlength{\marginparsep}{2em}%

% % using the factor of 1.62 (goldensection), 32 pica, 12pt only!!
\areaset[current]{384pt}{712pt}%   384*1.62 (goldensection) +  89.5 (headep+headheight+footskip), new calculation
% \areaset[current]{384pt}{697pt}%   384*1.62 (hexagon) +  75 (headep+headheight+footskip), old calculation
\setlength{\marginparwidth}{7em}%
\setlength{\marginparsep}{2em}%





% ********************************************************************
% Using different fonts
% ********************************************************************
%\usepackage[oldstylenums]{kpfonts} % oldstyle notextcomp
% \usepackage[libertine]{newtxmath}
% \usepackage[osf]{libertine}

%\usepackage[light,condensed,math]{iwona}
%\renewcommand{\sfdefault}{iwona}
%\usepackage{lmodern} % <-- no osf support :-(
%\usepackage{cfr-lm} %
%\usepackage[urw-garamond]{mathdesign} <-- no osf support :-(
%\usepackage[default,osfigures]{opensans} % scale=0.95
%\usepackage[sfdefault]{FiraSans}
% \usepackage[opticals,mathlf]{MinionPro} % onlytext
% ********************************************************************
% \usepackage[largesc,osf]{newpxtext} % smallcaps is 8% bigger with largesc
\usepackage[theoremfont,tighter,p,osf,largesc]{newpxtext}
\linespread{1.05} % a bit more for Palatino
\usepackage[T1]{fontenc}
% \usepackage{textcomp} % required for special glyphs 
\usepackage[bigdelims,vvarbb]{newpxmath} % too many math alphabets error
% \usepackage[scr=rsfso]{mathalfa}% \mathscr is fancier than \mathcal

\usepackage{sidenotes}


% Used to fix these:
% https://bitbucket.org/amiede/classicthesis/issues/139/italics-in-pallatino-capitals-chapter
% https://bitbucket.org/amiede/classicthesis/issues/45/problema-testatine-su-classicthesis-style
% ********************************************************************
% ****************************************************************************************************

%% microtype
% \usepackage{microtype}