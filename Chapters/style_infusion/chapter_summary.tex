In this chapter, we introduce a method to infuse audience-centric styles into pretrained language generation models. Specifically, we bootstrap an initial style analysis model to discriminate the positive and negative samples from audience feedback. Our model then selects additional samples from a generic topical sentence collection to expand the seed set of audience judgments. By separating style analysis and text generation models, we create an adversarial setup to infuse the audience's stylistic feedback in any generative LM. We weight the noisy reward from the style analysis model (discriminator) with a reconstruction loss to balance style adoption and fluency. We test our approach on two audience-centric styles, persuasion, and memorability, including empirical and qualitative evidence to show that our infusion approach generates compelling stylized examples with generic text prompts.\footnote{This work has led to the following publication: Samraj Moorjani, Adit Krishnan, Hari Sundaram, Ewa Maslowska, and Aravind Sankar. ``Audience-Centric Natural Language Generation via Style Infusion.'' \textit{Findings of the Association for Computational Linguistics: EMNLP 2022} \citep{moorjani-etal-2022-audience}.}

In summary, our abbreviated contributions are as follows: 
\begin{enumerate}
    \item \textbf{Audience-centric Style Infusion}: We introduce the task of style infusion to tether the definition of style to the target audience. By utilizing explicit audience feedback via pairwise comparisons, we promote a more human- and audience-centric approach to text styling.
    
    \item \textbf{Decoupling Style}: We propose an automatically weighted loss to decouple the training objectives of the style discriminator and text generation model, leading to a more robust representation of style than in fused settings.

    \item \textbf{Automatic Style Evaluation}: We present an automatic evaluation metric for quantifying the transfer of audience-centric styles by computing the correlations of linguistic features with a given style. We then evaluate our model's generations based on their agreements with the audience-derived correlations. 
\end{enumerate}