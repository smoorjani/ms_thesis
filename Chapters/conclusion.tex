% \section{Conclusion}
% p1 summarize findings p2 consequences/impact
This thesis investigates how to synthesize persuasive content through a combination of state-of-the-art natural language generation techniques and theoretical constructs from advertising and journalism. 

In~\Cref{chp:style_infusion}, we introduced \textit{style infusion} to motivate infusing audience-centric, stylistic preferences into unconstrained natural language generation models. We present a bootstrapped data augmentation method for limited pair-wise audience feedback and an adversarial training framework with a decoupling loss to train a style-infused GPT-2. Through an automatic evaluation method for the transfer of audience-specific styles, we show that our approach generates compelling stylized examples with generic text prompts better than the baselines. 

In~\Cref{chp:control_gen}, we experiment with novel approaches for controllable natural language generation to condition text on a "tuning-knob" for target controls (\eg speed). We present a modified prototype-then-edit model \citep{guu2018generating} and empirically show that our modifications lead to a substantial control on speed. We then alter our architecture from~\Cref{chp:style_infusion} to adversarially enforce a desired change in speed, finding that it can control speed, although less strongly than the previous approach.

Synthesizing text with highly subjective styles, such as persuasion and memorability, remains a significant challenge in domains like computational advertising. Our work takes the first few steps to address this problem, and we hope to inspire future research on this topic with this thesis. Ultimately, we aspire for the persuasive text generation systems we develop to be used with abstractive summarization techniques for scientific literature to generate more approachable and appealing science communication.

We mention the limitations of our work in~\Cref{sec:limitations} and then address the ethical implications of this thesis in~\Cref{sec:ethics}.

\section{Limitations}
\label{sec:limitations}

As with other unconstrained natural language generation applications, our systems are prone to issues like degeneration from beam search and neural hallucinations \citep{ji2022survey}. We post-process generations to combat the former problem, but future work will hopefully provide better methods to prevent this issue. The latter primarily applies to~\Cref{chp:style_infusion}, in which we increase our dataset with samples from the CNN/DailyMail dataset \citep{see-etal-2017-get}, partially mitigating the problem, but out-of-domain topics still suffer. Expanding the dataset size will only work for so long due to diminishing marginal returns.

We also note that our frameworks are limited by the computational resources available to us. Thus, we could not effectively support long text generation while preserving the quality of the generated text. During training, we decreased the batch size and utilized the DeepSpeed framework \citep{rasley-2020-deepspeed}, but it is still insufficient to handle long text. Furthermore, traditional left-to-right generation struggles with long text as the topics tend to diverge \citep{Tan2020}. Because many styles, like persuasiveness, depend on paragraph-level features in addition to sentence-level features, it is beneficial for our application to support longer texts.

Due to the limited data available for our style infusion approach, we considered iteratively training the discriminator with the augmented data while we trained the generator. Ultimately, we felt that the weak labels would dilute the learned trends in the discriminator, but it may be interesting to see how it affects the framework's performance. Currently, collecting pairwise datasets to use with this framework can be considered a limitation. With increasing interest in the computational synthesis of persuasive text and imagery, we expect to see more relevant curated datasets in the near future. Generating pairwise data through human subject experiments is expensive, which is why the data augmentation methods introduced in this paper are crucial for future work. We also note that while we claim that humans are better at pairwise evaluations, recent work has shown that humans are not always the gold standard for labeling, especially in natural language generation tasks \citep{clark2021all}.

One of the most significant limitations of the style infusion approach is in showing the effectiveness of our chosen architecture. Because most baselines are in style transfer and fundamentally differ from our task, we find it difficult to make a fair comparison with prior work. Regardless, style infusion is a critical step for unconstrained NLG systems such as dialogue systems and chatbots, especially in the context of human-centric stylistic objectives, which are already difficult enough to define. We encounter similar challenges in our work on stylized controllable text generation because many existing architectures do not generalize well and produce poor results. 

As mentioned in~\Cref{subsec:ne_results}, the modified prototype-then-edit model for controllable text generation is limited by its formulation. While it produces good results for speed, it will likely struggle with other constraints due to how the edit vector is constructed and requires expensive retraining for different values of $\Delta$. Conversely, the adversarial architecture in~\Cref{sec:adversarial_control} is likely able to adapt well to other constraints and can exhibit control on any arbitrary value of $\Delta$. However, neither of our current formulations would allow for the control of multiple constraints simultaneously, eliminating them as candidates for controlling styles that depend on the composition of features (\eg persuasiveness). 






\section{Ethics Statement}
\label{sec:ethics}

Our objective for developing controllable stylistic generative language models that leverages domain and audience-specific feedback is to enable unconstrained generation applications to appeal to more human users. For example, generating more persuasive real news might help combat misinformation by propagating the truth faster than falsehoods. In advertising and communication, persuasiveness and memorability are critical traits, and having an unconstrained generation model that could replicate these features would have a multitude of positive applications, especially in targeted interventions. Previous research has focused chiefly on predicting audience characteristics and targeting rather than synthesizing matching messages.

We acknowledge the dual-use concerns of the misuse of such a generation framework to, for example, spread misinformation. For this reason, we do not release the models or the pretrained generator checkpoints used in this work. 



