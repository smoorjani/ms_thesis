%*******************************************************
% Abstract
%*******************************************************
%\renewcommand{\abstractname}{Abstract}
\pdfbookmark[1]{Abstract}{Abstract}
% \addcontentsline{toc}{chapter}{\tocEntry{Abstract}}
\begingroup
\let\clearpage\relax
\let\cleardoublepage\relax
\let\cleardoublepage\relax

\chapter*{Abstract}
Adopting contextually appropriate, audience-tailored linguistic styles, namely persuasiveness, is critical to the success of user-centric language generation systems (\eg chatbots, computer-aided writing, dialog systems). While existing approaches demonstrate textual style transfer with large volumes of data, grounding style on audience-independent factors is innately limiting because many stylistic objectives (\eg persuasiveness, memorability, empathy) are hard to define without audience feedback.

In this thesis, we first propose the novel task of \textit{style infusion} - infusing the stylistic preferences of audiences in pretrained language generation models. Since humans are better at pairwise comparisons than direct scoring - \ie \textit{is Sample-A more persuasive than Sample-B} - we leverage limited pairwise human judgments to bootstrap a style analysis model and augment our seed set of judgments. We infuse the learned textual style in a GPT-2 based text generator while balancing fluency and style adoption. With quantitative and novel qualitative assessments, we show that our infusion approach can generate compelling stylized examples with generic text prompts. 

We then utilize complex linguistic features strongly correlated with persuasiveness to guide the generation of sequence-to-sequence models. We explore modifications of two approaches - an edit-then-prototype model and the style infusion architecture - to exhibit a ``tuning-knob''-like control over the speed of text (\ie how quickly content is covered). We empirically show that our modifications lead to strong controls over generated text and discuss directions to improve the fluency and control of generations further.

\vfill

\endgroup

\vfill
